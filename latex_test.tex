\documentclass[a4paper]{IEEEconf}

\begin{document}
\title{Uma nova abordagem para o planejador de movimentos RRT e para o algoritmo de campos potenciais artificiais -- aplicações de desvio de obstáculos em ambientes dinâmicos}

\thanks[footnoteinfo]{Trabalho financiado em parte pelo CNPq e em parte pela FAPES - Fundação de Amparo à Pesquisa e Inovação do Espírito Santo.}

\author{Luiz Miguel Monteiro Nascimento Pessotti Tavares,}
\author{Mario Sarcinelli-Filho,}
\author[Fourth]{Daniel Khéde Dourado Villa}

\address[Fourth]{Departamento de Engenharia Elétrica,
	Universidade Federal do Espírito Santo, ES, Brasil, (e-mail: daniel.villa@ufes.br)}

\selectlanguage{english}
\renewcommand{\abstractname}{{\bf Abstract:~}}
\begin{abstract}
	A motion planner is proposed to guide a wheeled mobile robot from initial to final positions. A conventional RRT (Rapidly-exploring Random Tree) algorithm is used for global planning, generating a non-optimal path for the entire route. Additionally, a second RRT algorithm is executed locally within a
	4 m$^2$ window with a resolution of \SI{5}{cm}. This local planning ensures that the system reacts to new obstacles that may arise along the robot's path and smooths the path generated by the global RRT. Complementing the navigation, an obstacle avoidance algorithm based on artificial potential fields is also proposed, ensuring that the system avoids collisions even under disturbances or parametric uncertainties. Experimental results and presented simulations validate the proposal.
\end{abstract}
\end{document}
